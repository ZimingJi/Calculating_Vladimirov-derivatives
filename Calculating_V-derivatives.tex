\documentclass[12pt]{article}
 
\usepackage[margin=1in]{geometry}
\usepackage{amsmath,amsthm,amssymb}
\usepackage{url}
 
\begin{document}
 
\title{Calculating Vladimirov derivatives and a little more}
\author{Ziming Ji}
 
\maketitle
 
\section{Derivative of the $\chi$ character}
Calculating Vladimirov derivative of $\chi(k_ny)$ where $k_n=p^{-n}$:
\begin{equation}
\begin{aligned}
D^s_y \chi(k_n y)&=\frac{1}{\Gamma_p (-s)} \int\limits _{\mathbb{Q}_p}dx \frac{\chi( p^{-n} x) - \chi( p^{-n}y)}{|x-y|_p^{1+s}}\\
&=\frac{1}{\Gamma_p (-s)} \left(\int\limits _{\mathbb{Q}_p}du \frac{\chi( p^{-n} (u+y))}{|u|_p^{1+s}}-\int\limits _{\mathbb{Q}_p}du \frac{\chi( p^{-n} y)}{|u|_p^{1+s}}\right)\\
&=\frac{1}{\Gamma_p (-s)} \chi( p^{-n} y)\left(\int\limits _{\mathbb{Q}_p}du \frac{\chi( p^{-n} u)}{|u|_p^{1+s}}-\int\limits _{\mathbb{Q}_p}du \frac{1}{|u|_p^{1+s}}\right)\\
&=\frac{1}{\Gamma_p (-s)} \chi( p^{-n} y)\left(-p^{(n-1)(s+1)-n}+\sum\limits_n^\infty p^{ls}(1-\frac{1}{p})-\sum\limits_{-\infty}^\infty p^{ls}(1-\frac{1}{p})\right)\\
&=\frac{1}{\Gamma_p (-s)} \chi( p^{-n} y)\left(p^{ns}\Gamma_p(-s)\right)=p^{ns}\chi(k_n y)
\end{aligned}
\end{equation}
We notice that two ways of regulating the integral are equivalent:
\begin{itemize}
\item Including $(- \chi( p^{-n}y))$ in the numerator;
\item ``Analytic continuation" of $\Gamma_p(-s)=\int\limits_{\mathbb{Q}_p}\frac{dx}{|x|_p}\chi(x)|x|_p^{-s}$ to the region $-s>0$ by $\Gamma_p(s)=\frac{\zeta_p(s)}{\zeta_p(1-s)}$.
\end{itemize}
It is really like defining all the quantities modulo $\sum\limits_{-\infty}^\infty p^{ls}$.
\section{Is V derivative equipped with a Leibniz rule and/or a product rule?}
Let's find the answer.

Sadly, there is no general ordinary product rule. But a reduced version comes out.
\begin{equation}
\begin{aligned}[t]

\end{aligned}
\end{equation}

What about a Leibniz rule?
\section{Check shift invariance of a general $p$-adic integral}
\subsection{Integrand depending only on the $p$-adic norm}
Define a general integral over $p$-adic number field as:
\begin{equation}
\begin{aligned}
&\quad \int\limits_{\mathbb{Q}_p} dx f(|x|_p)=\sum\limits_{l=-\infty}^{\infty}\int\limits_{p^l\mathbb{U}_p}dx f(p^{-l})=\sum\limits_{l=-\infty}^{\infty}f(p^{-l})(1-\frac{1}{p})p^{-l}
\end{aligned}
\end{equation}
A shift of the integral variable gives:
\begin{equation}
\begin{aligned}
&\quad \int\limits_{\mathbb{Q}_p} dx f(|x|_p)=\int\limits_{\mathbb{Q}_p} d(u-y) f(|u-y|_p) \quad\text{where}\quad u=x+y \\
\end{aligned}
\end{equation}
Suppose that $d(u-y)=du$ when $y$ is constant. Then we can proceed by splitting $\mathbb{Q}_p$:
\begin{equation}
\begin{aligned}
&\quad \int\limits_{\mathbb{Q}_p} d(u) f(|u-y|_p)=\sum\limits_{l=-\infty}^{\infty}\int\limits_{p^l\mathbb{U}_p}du f(|u-y|_p)
\end{aligned}
\end{equation}
Note that we cannot write $|u-y|_p$ as $p^{-l}$ in each block any more. Suppose that $y=p^m(a_0+a_1 p+a_2 p^2+...)$, we have:
\begin{equation}
\begin{aligned}
\sum\limits_{l=-\infty}^{\infty}\int\limits_{p^l\mathbb{U}_p}du f(|u-y|_p)&=\sum\limits_{l=-\infty}^{m-1}\int\limits_{p^l\mathbb{U}_p}du f(p^{-l})+\sum\limits_{l=m+1}^{\infty}\int\limits_{p^l\mathbb{U}_p}du f(p^{-m})+(l=m)\text{ term}\\
&=\sum\limits_{l=-\infty}^{m-1}f(p^{-l})(1-\frac{1}{p})p^{-l}+f(p^{-m})p^{-(m+1)}+(l=m)\text{ term}
\end{aligned}
\end{equation}
This ``contact term" can be written as:
\begin{equation}
\begin{aligned}
&\quad\int\limits_{p^m\mathbb{U}_p}du f(|u-y|_p)=\sum\limits_{b_0=1,\neq a_0}^{p-1}\int\limits_{p^{m+1}\mathbb{Z}_p}du f(p^{-m})+(a_0=b_0)\text{ term}\\
&=f(p^{-m})(p-2)p^{-(m+1)}+\sum\limits_{b_1=0,\neq a_1}^{p-1}\int\limits_{p^{m+2}\mathbb{Z}_p}du f(p^{-(m+1)})+(a_1=b_1)\text{ term}\\
&=f(p^{-m})(p-2)p^{-(m+1)}+\sum\limits_{l=m+1}^{\infty}f(p^{-l})(1-\frac{1}{p}) p^{-l}
\end{aligned}
\end{equation}
In the end, all these terms add up to the integral after shift:
\begin{equation}
\begin{aligned}
f(p^{-m})p^{-(m+1)}(p-1)+\sum\limits_{l=-\infty}^{m-1}f(p^{-l})(1-\frac{1}{p})p^{-l}+\sum\limits_{l=m+1}^{\infty}f(p^{-l})(1-\frac{1}{p}) p^{-l}
\end{aligned}
\end{equation}
This is exactly $\sum\limits_{l=-\infty}^{\infty}f(p^{-l})(1-\frac{1}{p})p^{-l}$, the integral before shifting.
\subsection{Integrand depending also on $p$-adic digits}
\section{Useful Fourier(and maybe Mellin) transforms}
A most common one is the Fourier transform of $|k|_p^s$ or a Mellin transform of $\chi(kx)$:
\begin{equation}
\int\limits_{\mathbb{Q}_p} dk \chi(kx)|k|_p^s=\Gamma_p(s+1)\frac{1}{|x|_p^{s+1}},
\end{equation}
recalling that the definition of $\Gamma_p$ is:
\begin{equation}
\Gamma_p(s)=\int\limits_{\mathbb{Q}_p}\frac{du}{|u|}\chi(u)|u|^s.
\end{equation}
A slight variation is:
\begin{equation}
\int\limits_{\mathbb{Q}_p} dk \chi(kx)|k|_p^s\left(\frac{k}{p}\right)_L=up^{s+\frac{1}{2}}\frac{\left(\frac{x}{p}\right)_L}{|x|_p^{s+1}},
\end{equation}
where $\left(\frac{k}{p}\right)_L$ is the Legendre symbol, $u$ equals $1$ for $p=1 \text{ mod } 4$ while $i$ for $p=3 \text{ mod } 4$. A useful Gauss sum here is
\begin{equation}
\sum\limits_{\alpha\in \mathbb{F}_p}\left(\frac{k}{p}\right)_L \chi\left(-\frac{k_0 \alpha}{p}\right)=\frac{\sqrt{p}}{u}\left(\frac{k_0}{p}\right)_L.
\end{equation}
We want to know about the Fourier transform of the characteristic function itself:
\begin{equation}
\int\limits_{\mathbb{Q}_p} dk \chi(kx)\chi(ky)=\frac{1}{|x+y|}\Gamma_p(1).
\end{equation}
\subsection{A simple example of Adelic product relation}
We need to know the Mellin transform of $e^{-\pi x^2}$ in $\mathbb{R}$ and $\gamma_0(x)$ the indicator(or characteristic) function of $\mathbb{Z}_p$ in their respective fields:
\begin{equation}
\int\limits_\mathbb{R}\frac{dx}{|x|}|x|^s e^{-\pi x^2}=\pi^{-\frac{s}{2}}\Gamma(\frac{s}{2})\equiv\zeta_\infty(s), \quad \int\limits_{\mathbb{Q}_p}\frac{dx}{|x|_p}|x|_p^s \gamma_0(x)=(1-\frac{1}{p})(\frac{1}{1-p^{-s}})\equiv(1-\frac{1}{p})\zeta_p(s)
\end{equation}
and the Mellin transform of respective character functions are definitions of the Gelfan-Graev gamma functions:
\begin{equation}
\int\limits_\mathbb{R}dx|x|^s e^{2\pi i x}\equiv\Gamma_\infty(s), \quad \int\limits_{\mathbb{Q}_p}\frac{dx}{|x|_p}|x|_p^s \chi(x)\equiv\Gamma_p(s).
\end{equation}
Interestingly, we have
\begin{equation}
\Gamma_\infty(s)=\frac{\zeta_\infty(s)}{\zeta_\infty(1-s)}, \quad \Gamma_p(s)=\frac{\zeta_p(s)}{\zeta_p(1-s)}.
\end{equation}
And remember that the definition of completed Riemann Zeta function $\xi(s)$ is an Adelic Mellin transform of the indicator functions($e^{-\pi x^2}$ is an analogue in Reals?) over Haar measures, which cancels the $(1-\frac{1}{p})$ factor in the integral over $\mathbb{Q}_p$. This is
\begin{equation}
\xi(s)=\zeta_\infty(s)\prod\limits_p\zeta_p(s)=\pi^{-\frac{s}{2}}\Gamma(\frac{s}{2})\zeta(s).
\end{equation}
This completed Riemann Zeta function solves the functional equation
\begin{equation}
\xi(s)=\xi(1-s).
\end{equation}
Then
\begin{equation}
\Gamma_\infty(s)\prod\limits_p\Gamma_p(s)=\frac{\zeta_\infty(s)}{\zeta_\infty(1-s)}\prod\limits_p\frac{\zeta_p(s)}{\zeta_p(1-s)}=\frac{\xi(s)}{\xi(1-s)}=1.
\end{equation}
Note that indicator functions($e^{-\pi x^2}$ is an analogue in Reals?) are their own Fourier transforms.
\section{Gauss integral}
\begin{equation}
\int\limits_{\mathbb{Q}_p}\chi(\alpha x^2+\beta x)dx=\lambda_p(\alpha)|2\alpha|_p^{-\frac{1}{2}}\chi(\frac{\beta^2}{4\alpha}), \quad \alpha\neq 0,
\end{equation}
where $\lambda_p(\alpha)$ is an arithmetic complex-valued function \cite{Vladimirov:1994zz}.

\newpage
\bibliographystyle{utphys}
\bibliography{Calculating_V-derivatives}
\end{document}

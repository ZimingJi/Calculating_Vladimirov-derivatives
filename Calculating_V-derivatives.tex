\documentclass[12pt]{article}
 
\usepackage[margin=1in]{geometry}
\usepackage{amsmath,amsthm,amssymb}
 
\newcommand{\N}{\mathbb{N}}
\newcommand{\R}{\mathbb{R}}
\newcommand{\Z}{\mathbb{Z}}
\newcommand{\Q}{\mathbb{Q}}
 
\newenvironment{theorem}[2][Theorem]{\begin{trivlist}
\item[\hskip \labelsep {\bfseries #1}\hskip \labelsep {\bfseries #2.}]}{\end{trivlist}}
\newenvironment{lemma}[2][Lemma]{\begin{trivlist}
\item[\hskip \labelsep {\bfseries #1}\hskip \labelsep {\bfseries #2.}]}{\end{trivlist}}
\newenvironment{exercise}[2][Exercise]{\begin{trivlist}
\item[\hskip \labelsep {\bfseries #1}\hskip \labelsep {\bfseries #2.}]}{\end{trivlist}}
\newenvironment{problem}[2][Problem]{\begin{trivlist}
\item[\hskip \labelsep {\bfseries #1}\hskip \labelsep {\bfseries #2.}]}{\end{trivlist}}
\newenvironment{question}[2][Question]{\begin{trivlist}
\item[\hskip \labelsep {\bfseries #1}\hskip \labelsep {\bfseries #2.}]}{\end{trivlist}}
\newenvironment{corollary}[2][Corollary]{\begin{trivlist}
\item[\hskip \labelsep {\bfseries #1}\hskip \labelsep {\bfseries #2.}]}{\end{trivlist}}
 
\begin{document}
 
\title{Calculating Vladimirov derivatives}
\author{Ziming Ji}
 
\maketitle
 
\section{Derivative of the $\chi$ character}
Calculating Vladimirov derivative of $\chi(k_ny)$ where $k_n=p^{-n}$:
\begin{equation}
\begin{aligned}
D^s_y \chi(k_n y)&=\frac{1}{\Gamma_p (-s)} \int\limits _{\mathbb{Q}_p}dx \frac{\chi( p^{-n} x) - \chi( p^{-n}y)}{|x-y|_p^{1+s}}\\
&=\frac{1}{\Gamma_p (-s)} \left(\int\limits _{\mathbb{Q}_p}du \frac{\chi( p^{-n} (u+y))}{|u|_p^{1+s}}-\int\limits _{\mathbb{Q}_p}du \frac{\chi( p^{-n} y)}{|u|_p^{1+s}}\right)\\
&=\frac{1}{\Gamma_p (-s)} \chi( p^{-n} y)\left(\int\limits _{\mathbb{Q}_p}du \frac{\chi( p^{-n} u)}{|u|_p^{1+s}}-\int\limits _{\mathbb{Q}_p}du \frac{1}{|u|_p^{1+s}}\right)\\
&=\frac{1}{\Gamma_p (-s)} \chi( p^{-n} y)\left(-p^{(n-1)(s+1)-n}+\sum\limits_n^\infty p^{ls}(1-\frac{1}{p})-\sum\limits_{-\infty}^\infty p^{ls}(1-\frac{1}{p})\right)\\
&=\frac{1}{\Gamma_p (-s)} \chi( p^{-n} y)\left(p^{ns}\Gamma_p(-s)\right)=p^{ns}\chi(k_n y)
\end{aligned}
\end{equation}
We notice that two ways of regulating the integral are equivalent:
\begin{itemize}
\item Including $(- \chi( p^{-n}y))$ in the numerator;
\item ``Analytic continuation" of $\Gamma_p(-s)=\int\limits_{\mathbb{Q}_p}\frac{dx}{|x|_p}\chi(x)|x|_p^{-s}$ to the region $-s>0$ by $\Gamma_p(s)=\frac{\zeta_p(s)}{\zeta_p(1-s)}$.
\end{itemize}
It is really like defining all the quantities modulo $\sum\limits_{-\infty}^\infty p^{ls}$.
\section{Check shift invariance of a general $p$-adic integral}
\subsection{Integrand depending only on the $p$-adic norm}
Define a general integral over $p$-adic number field as:
\begin{equation}
\begin{aligned}
&\quad \int\limits_{\mathbb{Q}_p} dx f(|x|_p)=\sum\limits_{l=-\infty}^{\infty}\int\limits_{p^l\mathbb{U}_p}dx f(p^{-l})=\sum\limits_{l=-\infty}^{\infty}f(p^{-l})(1-\frac{1}{p})p^{-l}
\end{aligned}
\end{equation}
A shift of the integral variable gives:
\begin{equation}
\begin{aligned}
&\quad \int\limits_{\mathbb{Q}_p} dx f(|x|_p)=\int\limits_{\mathbb{Q}_p} d(u-y) f(|u-y|_p) \quad\text{where}\quad u=x+y \\
\end{aligned}
\end{equation}
Suppose that $d(u-y)=du$ when $y$ is constant. Then we can proceed by splitting $\mathbb{Q}_p$:
\begin{equation}
\begin{aligned}
&\quad \int\limits_{\mathbb{Q}_p} d(u) f(|u-y|_p)=\sum\limits_{l=-\infty}^{\infty}\int\limits_{p^l\mathbb{U}_p}du f(|u-y|_p)
\end{aligned}
\end{equation}
Note that we cannot write $|u-y|_p$ as $p^{-l}$ in each block any more. Suppose that $y=p^m(a_0+a_1 p+a_2 p^2+...)$, we have:
\begin{equation}
\begin{aligned}
\sum\limits_{l=-\infty}^{\infty}\int\limits_{p^l\mathbb{U}_p}du f(|u-y|_p)&=\sum\limits_{l=-\infty}^{m-1}\int\limits_{p^l\mathbb{U}_p}du f(p^{-l})+\sum\limits_{l=m+1}^{\infty}\int\limits_{p^l\mathbb{U}_p}du f(p^{-m})+(l=m)\text{ term}\\
&=\sum\limits_{l=-\infty}^{m-1}f(p^{-l})(1-\frac{1}{p})p^{-l}+f(p^{-m})p^{-(m+1)}+(l=m)\text{ term}
\end{aligned}
\end{equation}
This ``contact term" can be written as:
\begin{equation}
\begin{aligned}
&\quad\int\limits_{p^m\mathbb{U}_p}du f(|u-y|_p)=\sum\limits_{b_0=1,\neq a_0}^{p-1}\int\limits_{p^{m+1}\mathbb{Z}_p}du f(p^{-m})+(a_0=b_0)\text{ term}\\
&=f(p^{-m})(p-2)p^{-(m+1)}+\sum\limits_{b_1=0,\neq a_1}^{p-1}\int\limits_{p^{m+2}\mathbb{Z}_p}du f(p^{-(m+1)})+(a_1=b_1)\text{ term}\\
&=f(p^{-m})(p-2)p^{-(m+1)}+\sum\limits_{l=m+1}^{\infty}f(p^{-l})(1-\frac{1}{p}) p^{-l}
\end{aligned}
\end{equation}
In the end, all these terms add up to the integral after shift:
\begin{equation}
\begin{aligned}
f(p^{-m})p^{-(m+1)}(p-1)+\sum\limits_{l=-\infty}^{m-1}f(p^{-l})(1-\frac{1}{p})p^{-l}+\sum\limits_{l=m+1}^{\infty}f(p^{-l})(1-\frac{1}{p}) p^{-l}
\end{aligned}
\end{equation}
This is exactly $\sum\limits_{l=-\infty}^{\infty}f(p^{-l})(1-\frac{1}{p})p^{-l}$, the integral before shifting.
\subsection{Integrand depending also on $p$-adic digits}
\end{document}

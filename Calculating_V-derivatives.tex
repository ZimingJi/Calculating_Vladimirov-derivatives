\documentclass[12pt]{article}
 
\usepackage[margin=1in]{geometry}
\usepackage{amsmath,amsthm,amssymb}
 
\newcommand{\N}{\mathbb{N}}
\newcommand{\R}{\mathbb{R}}
\newcommand{\Z}{\mathbb{Z}}
\newcommand{\Q}{\mathbb{Q}}
 
\newenvironment{theorem}[2][Theorem]{\begin{trivlist}
\item[\hskip \labelsep {\bfseries #1}\hskip \labelsep {\bfseries #2.}]}{\end{trivlist}}
\newenvironment{lemma}[2][Lemma]{\begin{trivlist}
\item[\hskip \labelsep {\bfseries #1}\hskip \labelsep {\bfseries #2.}]}{\end{trivlist}}
\newenvironment{exercise}[2][Exercise]{\begin{trivlist}
\item[\hskip \labelsep {\bfseries #1}\hskip \labelsep {\bfseries #2.}]}{\end{trivlist}}
\newenvironment{problem}[2][Problem]{\begin{trivlist}
\item[\hskip \labelsep {\bfseries #1}\hskip \labelsep {\bfseries #2.}]}{\end{trivlist}}
\newenvironment{question}[2][Question]{\begin{trivlist}
\item[\hskip \labelsep {\bfseries #1}\hskip \labelsep {\bfseries #2.}]}{\end{trivlist}}
\newenvironment{corollary}[2][Corollary]{\begin{trivlist}
\item[\hskip \labelsep {\bfseries #1}\hskip \labelsep {\bfseries #2.}]}{\end{trivlist}}
 
\begin{document}
 
\title{Calculating Vladimirov derivatives}
\author{Ziming Ji}
 
\maketitle
 
\begin{section}{Derivative of the $\chi$ character}
It seems to me that there are two ways to do the calculation: one is to split $\mathbb{Q}_p$ into different domains($\xi \mathbb{U}_p$) and do the integral of $\chi$; the other is to decompose the $\chi$ character function into piece-wise constant functions $\gamma_n$ and apply the V-derivatives on them.

\begin{paragraph}{a) Splitting $\mathbb{Q}_p$}

\begin{equation}
D^s_y \chi(k_n y)=\frac{1}{\Gamma_p (-s)} \int\limits _{\mathbb{Q}_p}dx \frac{\chi( p^{-n} x) - \chi( p^{-m}y_0)}{|x-p^{n-m}y_0|_p^{1+s}}
\end{equation}
Where $k_n=p^{-n}$ and $y=p^{n-m} y_0$ with $m\geq 1$(other wise $\chi$ will be a constant function and the derivative is zero).\\
Useful integrals are:
\begin{equation}
\int\limits_{\xi \mathbb{U}_p}dx\chi(x)=|\xi|_p(\gamma_0(\xi)-\frac{1}{p}\gamma_0(p\xi))
=\begin{cases}
|\xi|_p(1-\frac{1}{p}), &v_p(\xi)\geq 0\\
-1,  &v_p(\xi) = -1\\
0,  &v_p(\xi)\leq -1
\end{cases}
\end{equation}
Consider the integral in each $x\in p^l \mathbb{U}_p$ block($x=p^l x_0$ where $x_0\in \mathbb{U}_p$) and we have 3 parts:\\
First, for $l<n-m$, we have $|x-y|_p=|x|_p=p^{-l}$ and the integral becomes:
\begin{equation}
\begin{split}
&\quad\frac{1}{\Gamma_p(-s)}\sum_{l=-\infty}^{n-m-1} \int\limits _{p^l \mathbb{U}_p}dx \frac{\chi( p^{-n} x) - \chi( p^{-m}y_0)}{p^{-l(1+s)}}\\\
&= \frac{1}{\Gamma_p(-s)}\sum_{l=-\infty}^{n-m-1} \int\limits _{p^{l-n} \mathbb{U}_p}dz |p^n|_p \frac{\chi(z) - \chi( p^{-m}y_0)}{p^{-l(1+s)}} \\
&= \frac{1}{\Gamma_p(-s)}\sum_{l=-\infty}^{n-m-1} p^{-n} \frac{0 - \chi( p^{-m}y_0)}{p^{-l(1+s)}} (1-\frac{1}{p}) |p^{l+n}|_p \\
&= \frac{1}{\Gamma_p(-s)}\sum_{l=-\infty}^{n-m-1} - \chi( p^{-m}y_0)(1-\frac{1}{p})p^{ls}\\
&= \frac{- \chi( p^{-m}y_0)}{\Gamma_p(-s)} \frac{p-1}{p^s-1} p^{s(n-m)-1}.
\end{split}
\end{equation}
The second part, $l=n-m$, is more complicated. Note that $\mathbb{U}_p=\{a_0\}+p\mathbb{Z}_p$, where $a_0$ is the first digit of numbers in $\mathbb{U}_p$. So we can do integrals in this way as well: 
\begin{equation}
\int\limits_{\mathbb{U}_p}dy=|\{a_0\}|\times\int\limits_{p\mathbb{Z}_p}dx=(p-1)|p|_p=1-\frac{1}{p}.
\end{equation} 
In this sense we can do the splitting infinitely as $p^{n-m}(\{a_0\}+\{a_1\}+\{a_2\}+\{a_3\}+...+\{a_{n-1}\}+p^n\mathbb{Z}_p)$, and the integral is equal to$(p-1)p^{n-1}|p^n|_p=1-\frac{1}{p}$ correctly. \\
\\Thus the integral region $p^{n-m}\mathbb{U}_p$ can be split into $p^{n-m}(\{a_0\}+\{a_1\}+\{a_2\}+\{a_3\}+...+\{a_{n-1}\}+p^n\mathbb{Z}_p)$. For a given $y_0\in\mathbb{U}_p$ and all $x_0\in\mathbb{U}_p$, there is 1 case where they have the same first digit(where $|x_0-y_0|_p$ is determined by following digits) and $p-2$ cases where they have different first digits(where $|x_0-y_0|_p=1$) ; there is 1 case where they have the same second digit(where $|x_0-y_0|_p$ is determined by following digits) and $p-1$ cases where they have different second digits(where $|x_0-y_0|_p=p^{-1}$); there is 1 case where they have the same third digit(where $|x_0-y_0|_p$ is determined by following digits) and $p-1$ cases where they have different third digits(where $|x_0-y_0|_p=p^{-2}$);...Continue this spiting till $n\to\infty$ and we should obtain the summation form of the integral. \\
\\Note that here different digits are weighted by $\chi(p^{-n}x)=\chi(p^{-m}x_0)$ and the summation is trickier. Consider for example for a \textbf{particular} first digit of $x_0$, $a_0$. Without any pre-factor and constant term, the integral is:
\begin{equation}
\begin{split}
\quad \int\limits_{p\mathbb{Z}_p}dx_0\frac{\chi(p^{-m}(a_0+a_1 p+...))}{|x_0-y_0|_p^{1+s}} \\
&=\int\limits_{\mathbb{Z}_p}dx_1\chi(p^{-m}a_0)\chi(p^{-m+1}x_1)\\
&=
\end{split}
\end{equation} 
The derivative is:
\begin{equation}
\begin{split}
&\quad\frac{1}{\Gamma_p (-s)} \int\limits _{p^{n-m}\mathbb{U}_p}dx\frac{\chi( p^{-n} x) - \chi( p^{-m}y_0)}{|p^{n-m}(x_0-y_0)|_p^{1+s}}\\
&=\frac{1}{\Gamma_p (-s)} \int\limits _{p^{n-m}\mathbb{U}_p}dx\frac{\chi( p^{-n} x) - \chi( p^{-m}y_0)}{|p^{n-m}(x_0-y_0)|_p^{1+s}}
\end{split}
\end{equation}
The last term, $l>n-m$, is:
\begin{equation}
\begin{split}
&\quad\frac{1}{\Gamma_p (-s)} \sum_{l=n-m+1}^{\infty} \int\limits _{p^l \mathbb{U}_p}dx\frac{\chi( p^{-n} x) - \chi( p^{-m}y_0)}{|p^{n-m}y_0|_p^{1+s}}\\
&=\frac{1}{\Gamma_p (-s)} p^{(n-m)(1+s)} \sum_{l=n-m+1}^{\infty} \int\limits _{p^l \mathbb{U}_p}dx (\chi( p^{-n} x) - \chi( p^{-m}y_0))\\
&=\frac{1}{\Gamma_p (-s)} p^{(n-m)(1+s)} (\sum_{l=n-m+1}^{\infty} \int\limits _{p^l \mathbb{U}_p}dx \chi( p^{-n} x) - \sum_{l=n-m+1}^{\infty} p^{-l}(1-\frac{1}{p})\chi( p^{-m}y_0))\\
&=\frac{1}{\Gamma_p (-s)} p^{(n-m)(1+s)} (\sum_{l=n-m+1}^{\infty} \int\limits _{p^l \mathbb{U}_p}dx \chi( p^{-n} x) - p^{-1+m-n}\chi( p^{-m}y_0))
\end{split}
\end{equation}
While this integral $\sum_{l=n-m+1}^{\infty}\int\limits _{p^l \mathbb{U}_p}dx \chi( p^{-n} x)$ should be considered in two cases: \\
\begin{itemize}
\item When $m=1$, this is $\sum^\infty_{l=n}p^l(1-\frac{1}{p})=p^{-n}$;
\item When $m\geq 2$, this is $-p^{-n}+\sum^\infty_{l=n}p^l(1-\frac{1}{p})=0$.
\end{itemize}
So in this region, the integral becomes:
\begin{equation}
\begin{cases}
\frac{1}{\Gamma_p (-s)} p^{(n-m)s-1} (p^{1-m} - \chi( p^{-m}y_0)), \quad m=1\\
\frac{-1}{\Gamma_p (-s)} p^{(n-m)s-1} \chi( p^{-m}y_0), \quad m\geq 2
\end{cases}
\end{equation}
\end{paragraph}

\end{section}

\end{document}

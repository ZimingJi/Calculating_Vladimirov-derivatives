\documentclass[12pt]{article}
 
\usepackage[margin=1in]{geometry}
\usepackage{amsmath,amsthm,amssymb}
\usepackage{tikz}
\usetikzlibrary{matrix,arrows}

 
\newcommand{\N}{\mathbb{N}}
\newcommand{\R}{\mathbb{R}}
\newcommand{\Z}{\mathbb{Z}}
\newcommand{\Q}{\mathbb{Q}}
 
\newenvironment{theorem}[2][Theorem]{\begin{trivlist}
\item[\hskip \labelsep {\bfseries #1}\hskip \labelsep {\bfseries #2.}]}{\end{trivlist}}
\newenvironment{lemma}[2][Lemma]{\begin{trivlist}
\item[\hskip \labelsep {\bfseries #1}\hskip \labelsep {\bfseries #2.}]}{\end{trivlist}}
\newenvironment{exercise}[2][Exercise]{\begin{trivlist}
\item[\hskip \labelsep {\bfseries #1}\hskip \labelsep {\bfseries #2.}]}{\end{trivlist}}
\newenvironment{problem}[2][Problem]{\begin{trivlist}
\item[\hskip \labelsep {\bfseries #1}\hskip \labelsep {\bfseries #2.}]}{\end{trivlist}}
\newenvironment{question}[2][Question]{\begin{trivlist}
\item[\hskip \labelsep {\bfseries #1}\hskip \labelsep {\bfseries #2.}]}{\end{trivlist}}
\newenvironment{corollary}[2][Corollary]{\begin{trivlist}
\item[\hskip \labelsep {\bfseries #1}\hskip \labelsep {\bfseries #2.}]}{\end{trivlist}}
 
\begin{document}
 
\title{Kaluza-Klein theory with ultrametric compact dimensions}
\author{Ziming Ji}
 
\maketitle
 
\section{Action}
\begin{equation} \label{action1}
S=\int\limits_{R^{3,1}}dx^4\int\limits_{\mathbb{Z}_p}dy\quad\frac{1}{2}\phi(\partial^2+D_y^s)\phi+V(\phi)
\end{equation}

\section{Consistent Truncation}
A certain type of dimensional reductions are consistent truncations:

$\mathbb{Z}_p\rightarrow p\mathbb{Z}_p\rightarrow p^2\mathbb{Z}_p\rightarrow ... \rightarrow 0$?

Two types of consistent truncation:
\begin{itemize}
\item Kaluza-Klein dimensional reduction without changing dof of each space-time point?
\item Introducing constraints that reduce independent dof.
\end{itemize}

What will happen if we truncate at $|k|_p=3$? Plugging in 
\begin{equation}
\phi(x)=\phi_0(x)+\phi _{-\frac{2}{3}}(x) \chi \left(-\frac{2 y}{3}\right)+\phi _{-\frac{1}{3}}(x) \chi \left(-\frac{y}{3}\right)+\phi _{\frac{1}{3}}(x) \chi \left(\frac{y}{3}\right)+\phi _{\frac{2}{3}}(x) \chi \left(\frac{2 y}{3}\right)
\end{equation}
to equation \ref{action1} and completing the $\mathbb{Z}_p$ integral, one obtains various terms:
\begin{itemize}
\item $\phi(x)\phi''(x)$ becomes
\begin{equation}
3^s \left(\phi _{\frac{2}{3}}(x)+\phi _{-\frac{1}{3}}(x)\right) \left(\phi _{\frac{1}{3}}''(x)+\phi _{-\frac{2}{3}}''(x)\right)+3^s \left(\phi _{\frac{1}{3}}(x)+\phi _{-\frac{2}{3}}(x)\right) \left(\phi _{\frac{2}{3}}''(x)+\phi _{-\frac{1}{3}}''(x)\right)+\phi _0(x) \phi _0''(x)
\end{equation}
We note $\left(\phi _{\frac{1}{3}}(x)+\phi _{-\frac{2}{3}}(x)\right)$ as $\phi_1$, then $\left(\phi _{\frac{2}{3}}(x)+\phi _{-\frac{1}{3}}(x)\right)$ is $\phi_1^*$(because $\phi$ is real). Then we have
\begin{equation}
\phi \partial^2 \phi =3^s (\phi_1^*\partial^2\phi_1+\phi_1 \partial^2\phi_1^*) + \phi _0\partial^2 \phi _0
\end{equation}
\item $\phi(x)^2$ becomes
\begin{equation}
\phi _0(x){}^2+2 \left(\phi _{\frac{1}{3}}(x)+\phi _{-\frac{2}{3}}(x)\right) \left(\phi _{\frac{2}{3}}(x)+\phi _{-\frac{1}{3}}(x)\right)=\phi _0(x){}^2+2 \phi_1 \phi_1^*
\end{equation}
\item $\phi(x)^3$ becomes
\begin{equation}
\phi _0^3+6 \phi _1 \left(\phi _1\right){}^* \phi _0+\phi _1^3+\left(\left(\phi _1\right){}^*\right){}^3
\end{equation}
\item $\phi(x)^4$ becomes
\begin{equation}
\phi _0^4+12 \phi _1 \left(\phi _1\right){}^* \phi _0^2+4 \left(\phi _1^3+\left(\left(\phi _1\right){}^*\right){}^3\right) \phi _0+6 \phi _1^2 \left(\left(\phi _1\right){}^*\right){}^2
\end{equation}
\item $\phi(x)^5$ becomes
\begin{equation}
\phi _0^5+20 \phi _1 \left(\phi _1\right){}^* \phi _0^3+10 \left(\phi _1^3+\left(\left(\phi _1\right){}^*\right){}^3\right) \phi _0^2+30 \phi _1^2 \left(\left(\phi _1\right){}^*\right){}^2 \phi _0+5 \phi _1 \left(\phi _1\right){}^* \left(\phi _1^3+\left(\left(\phi _1\right){}^*\right){}^3\right)
\end{equation}
\end{itemize}

\end{document}
